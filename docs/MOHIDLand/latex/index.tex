M\+O\+H\+ID is short for Modelo Hidrodinâmico which is hydrodynamic model in Portuguese. M\+O\+H\+ID is a three-\/dimensional water modelling system, developed by M\+A\+R\+E\+T\+EC (Marine and Environmental Technology Research Center) at Instituto Superior Técnico (I\+ST) which belongs to Lisbon University.

\subsection*{What is this repository?}

This is the M\+O\+H\+ID Water Modelling System O\+F\+F\+I\+C\+I\+AL repository

\subsection*{Overview}

M\+O\+H\+ID is a modular finite volumes water-\/modelling system written in A\+N\+S\+I-\/\+Fortran95 using an Object-\/oriented programming philosophy, integrating diverse mathematical models and supporting graphical user interfaces that manage all the pre-\/ and post-\/processing. M\+O\+H\+ID allows the adoption of an integrated modelling philosophy, not only of processes (physical and biogeochemical), but also of different scales (allowing the use of nested models) and systems (estuaries and watersheds), due to the adoption of an object oriented programming philosophy. The development of M\+O\+H\+ID started back in 1985. Since that time a continuous development effort of new features has been maintained. Model updates and improvements were made available in a regular basis were used in the framework of many research and engineering projects. All programs included in M\+O\+H\+ID Water Modelling System are built on the top of one or more base libraries and the two core executables files can be found at the top of the pyramid\+:
\begin{DoxyItemize}
\item M\+O\+H\+ID Water – Three-\/dimensional mathematical model to simulate surface water bodies.
\item M\+O\+H\+ID Land – Watershed mathematical model or Hydrological transport model designed to simulate drainage basin and aquifer;
\end{DoxyItemize}

Smaller utility programs are easily built on the top of the libraries, which are usually designed for pre or post-\/processing results of the models. This support tools are normally managed by graphical user interfaces which allow management of input data, control of program execution, and output results analysis, along with other pre-\/ and post-\/processing operations. The integration of M\+O\+H\+I\+D’s different tools can be easily achieved since these tools are based on the same framework. This coupling can thus be used to study the water cycle and its associated processes in an integrated approach.

\subsection*{Help, Bugs, Feedback}

If you need help with M\+O\+H\+ID, want to keep up with progress, chat with developers or ask any other questions about M\+O\+H\+ID, you can hang out by mail\+: \href{mailto:general@mohid.com}{\tt general@mohid.\+com} or consult our \href{http://wiki.mohid.com}{\tt M\+O\+H\+ID wiki}. You can also subscribe to our \href{http://forum.mohid.com}{\tt M\+O\+H\+ID forum}. To report bugs, please create a Git\+Hub issue or contact any developers. More information consult \href{http://www.mohid.com}{\tt http\+://www.\+mohid.\+com}

\subsection*{License}

G\+NU General Public License. See the \href{http://www.gnu.org/copyleft/gpl.html}{\tt G\+NU General Public License} web page for more information. 